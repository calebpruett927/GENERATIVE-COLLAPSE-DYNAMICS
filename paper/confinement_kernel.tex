% ============================================================
% Confinement as Integrity Collapse:
% A Measurable Structural Signature of Quark Binding
% in the Generative-Collapse Kernel
%
% Compile: pdflatex → bibtex → pdflatex → pdflatex
% ============================================================

\documentclass[
  aps,
  prd,
  twocolumn,
  superscriptaddress,
  nofootinbib,
  floatfix
]{revtex4-2}

% ── packages ──
\usepackage{amsmath,amssymb,amsthm}
\usepackage{graphicx}
\usepackage{hyperref}
\usepackage{xcolor}
\usepackage{bm}
\usepackage{booktabs}
\usepackage{multirow}

% ── theorem environments ──
\newtheorem{theorem}{Theorem}
\newtheorem{definition}[theorem]{Definition}
\newtheorem{lemma}[theorem]{Lemma}
\newtheorem{corollary}[theorem]{Corollary}
\newtheorem{proposition}[theorem]{Proposition}
\newtheorem{remark}{Remark}
\newtheorem{claim}{Claim}

% ── UMCP macros ──
\newcommand{\tR}{\tau_{\!R}^{*}}
\newcommand{\Gam}{\Gamma}
\newcommand{\eps}{\varepsilon}
\newcommand{\tolseam}{\mathrm{tol}_{\mathrm{seam}}}
\newcommand{\dd}{\mathrm{d}}
\newcommand{\beq}{\begin{equation}}
\newcommand{\eeq}{\end{equation}}
\newcommand{\INF}{\texttt{INF\_REC}}
\newcommand{\regime}[1]{\textsc{#1}}
\newcommand{\conform}{\regime{conformant}}
\newcommand{\tvec}{\mathbf{c}}
\newcommand{\wvec}{\mathbf{w}}
\newcommand{\GM}{\mathrm{GM}}
\newcommand{\AM}{\mathrm{AM}}
\newcommand{\IC}{\mathrm{IC}}
\newcommand{\LQCD}{\Lambda_{\mathrm{QCD}}}

\begin{document}

% ============================================================
% TITLE
% ============================================================
\title{Confinement as Integrity Collapse:\\
A Measurable Structural Signature of Quark Binding\\
in the Generative-Collapse Kernel}

\author{Caleb Pruett}
\email{caleb@umcp.dev}
\affiliation{UMCP Reference Implementation, GitHub}

\author{Clement Paulus}
\affiliation{GCD/UMCP Theoretical Framework}

\date{\today}

\begin{abstract}
We present a quantitative, domain-independent structural
signature of quark confinement derived from the Generative
Collapse Dynamics (GCD) kernel.  Without reference to the
QCD Lagrangian, gauge symmetry, or non-perturbative dynamics,
we show that mapping quarks and hadrons into 8-channel trace
vectors and computing the kernel integrity composite
$\IC = \exp(\kappa)$ produces a sharp cliff: $\IC$ drops by
$98.1\%$ at the quark$\to$hadron boundary.  All 14 tested
hadrons have $\IC$ below the minimum quark $\IC$.  The
heterogeneity gap $\Delta = F - \IC$ amplifies by $10.8\times$
upon binding.  We trace the mechanism to \emph{channel death}:
confinement annihilates specific observable channels
(strangeness, heavy flavor, charge, spin) in composite states,
and the geometric mean---being multiplicatively sensitive to
zeros---translates this annihilation into a two-order-of-magnitude
integrity collapse.  The result is formally proven as Theorem~T3
(19/19 sub-tests) within the GCD framework, derives independently
from Axiom-0 (``Collapse is generative; only what returns is
real''), and provides a measurable, falsifiable, computable
signature of confinement that does not require solving the
Yang-Mills mass gap problem.  All computations are reproducible
via the open-source reference implementation.
\end{abstract}

\maketitle

% ============================================================
\section{Introduction}\label{sec:intro}
% ============================================================

Quark confinement---the empirical fact that quarks are never
observed as free particles---is one of the deepest unsolved
problems in fundamental physics.  Despite five decades of effort,
no analytical proof exists that Quantum Chromodynamics (QCD)
confines quarks.  The problem is significant enough to constitute
one of the seven Clay Millennium Prize Problems: prove that the
$SU(3)$ Yang-Mills theory has a mass gap and
confines~\cite{jaffe2006quantum}.

The difficulty is intrinsically non-perturbative.  At high energies
(short distances), the strong coupling constant $\alpha_s$ becomes
small---this is asymptotic freedom~\cite{gross1973,politzer1973},
the 2004 Nobel Prize discovery.  Standard perturbative QFT
(Feynman diagrams, loop expansions, renormalization) works well
in this regime.  But as energy decreases toward the confinement
scale $\LQCD \approx 0.3$~GeV, $\alpha_s$ grows without bound,
and perturbation theory breaks down completely.  Lattice QCD
provides strong numerical evidence for confinement but has not
yielded a proof~\cite{wilson1974confinement}.

\textbf{What is known experimentally}: Every particle detector
ever built tells the same story.  Deep inelastic scattering at
SLAC (1968--1973) revealed point-like constituents inside
protons~\cite{pdg2024}.  Collider experiments at CERN, Fermilab,
and the LHC produce hadron jets but never isolated quarks.  When
sufficient energy is applied to separate quarks, the gluon flux
tube fragments into new $q\bar{q}$ pairs, producing additional
hadrons.  The evidence is universal: quarks bind into color-singlet
states ($qqq$ baryons, $q\bar{q}$ mesons) and nothing else
escapes.

\textbf{What this paper contributes}: We show that the GCD
kernel---a domain-agnostic measurement system derived from a
single axiom---produces a sharp, quantitative structural signature
of confinement \emph{without any knowledge of QCD}.  The kernel
receives normalized measurement vectors and computes five
invariants.  The structural signature is a $98.1\%$ collapse
in the integrity composite at the quark$\to$hadron boundary.
This result is:
\begin{enumerate}
\item \textbf{Domain-independent}: the same mechanism applies to
  any system where binding annihilates observable channels.
\item \textbf{Axiom-derived}: it follows from the mathematical
  properties of the integrity composite, which derives from
  Axiom-0.
\item \textbf{Falsifiable}: if any hadron were discovered with
  $\IC > \IC^{\min}_{\mathrm{quark}}$, the result would be
  refuted.
\item \textbf{Computable}: the entire analysis runs as automated
  tests in the reference implementation~\cite{umcpmetadatarepo}.
\end{enumerate}


% ============================================================
\section{The Scientific Mystery}\label{sec:mystery}
% ============================================================

\subsection{What QCD Says (But Cannot Prove)}

The QCD Lagrangian is
\beq\label{eq:qcd}
  \mathcal{L}_{\mathrm{QCD}} = \bar{\psi}_i
  (i\gamma^\mu D_\mu - m)\psi_i
  - \frac{1}{4} G^a_{\mu\nu} G^{a\mu\nu},
\eeq
where $\psi_i$ are quark fields carrying color index $i \in \{r,g,b\}$,
$D_\mu$ is the covariant derivative coupling quarks to the eight
gluon fields $A^a_\mu$ ($a = 1,\ldots,8$), and $G^a_{\mu\nu}$ is
the gluon field-strength tensor.

The key property of Eq.~\eqref{eq:qcd} is the running of the
strong coupling under renormalization:
\beq\label{eq:running}
  \alpha_s(Q^2) = \frac{\alpha_s(M_Z^2)}
    {1 + \frac{\alpha_s(M_Z^2)}{2\pi}
    (11 - \tfrac{2}{3}n_f) \ln\!\left(\frac{Q^2}{M_Z^2}\right)},
\eeq
where $n_f$ is the number of active flavors and $\alpha_s(M_Z) = 0.1180$
(PDG 2024~\cite{pdg2024}).  For $n_f \leq 16$ (realized as
$n_f = 6$ in nature), the coefficient $(11 - \frac{2}{3}n_f) > 0$,
so $\alpha_s$ decreases at high $Q$ (asymptotic
freedom~\cite{gross1973,politzer1973}) and grows at low $Q$.

As $Q \to \LQCD$, $\alpha_s$ diverges (the Landau pole),
signaling the complete breakdown of perturbation theory.  It is
in this non-perturbative regime that confinement is believed to
occur.  But ``believed'' is the operative word: no one has proven
analytically that the $SU(3)$ Yang-Mills theory confines quarks
and has a mass gap~\cite{jaffe2006quantum}.

\subsection{What Experiments Show}

The experimental evidence for confinement is overwhelming:

\begin{enumerate}
\item \textbf{Deep inelastic scattering} (Friedman, Kendall,
  Taylor --- Nobel 1990): protons have point-like constituents
  (quarks), but those constituents are never ejected as free
  particles.

\item \textbf{Jet fragmentation}: at the LHC ($\sqrt{s} = 14$~TeV),
  quarks produced in hard scattering form jets---collimated sprays
  of hadrons.  The quarks ``hadronize'' within $\sim 10^{-23}$~s.

\item \textbf{Quark-gluon plasma}: at RHIC and LHC heavy-ion
  collisions, quarks are briefly deconfined at temperatures
  $T > T_c \approx 150$~MeV, but re-confine as the plasma cools.

\item \textbf{No isolated quarks}: 50+ years of searching in
  cosmic rays, accelerator debris, and matter searches have found
  zero free quarks carrying fractional electric charge.
\end{enumerate}

\subsection{The Open Problem}

The Millennium Problem asks: prove that for any compact simple
gauge group $G$, quantum Yang-Mills theory on $\mathbb{R}^4$
exists (as a QFT satisfying Wightman axioms) and has a mass gap
$\Delta > 0$.  The mass gap implies confinement: color-charged
objects cannot propagate as free particles because the lightest
state in the spectrum has $m > 0$.

Lattice QCD (Wilson, 1974~\cite{wilson1974confinement}) provides
numerical evidence by discretizing spacetime and computing the
path integral on a grid.  The Wilson loop criterion $\langle W(C)
\rangle \sim \exp(-\sigma A)$ (where $\sigma$ is the string
tension and $A$ the minimal area enclosed by $C$) is satisfied
in lattice simulations.  But this is evidence, not proof.


% ============================================================
\section{Kernel Framework}\label{sec:kernel}
% ============================================================

The GCD kernel is an axiomatic measurement system derived from
a single axiom~\cite{paulus2025umcp,paulus2025episteme}:

\begin{quote}
\textbf{Axiom-0}: \textit{Collapse is generative; only what
returns is real.}
\end{quote}

\begin{definition}[Trace Vector]\label{def:trace}
A \emph{trace vector} is a measurement
$\tvec = (c_1, \ldots, c_n) \in [\eps, 1-\eps]^n$ with
associated weights $\wvec = (w_1, \ldots, w_n)$,
$\sum w_i = 1$, and guard band $\eps > 0$.  Each $c_i$ is
a normalized measurable property of the system.
\end{definition}

\begin{definition}[Kernel Invariants]\label{def:invariants}
Given $(\tvec, \wvec, \eps)$, the kernel computes:
\begin{align}
  F   &= \sum_{i=1}^n w_i\, c_i
    & &\text{(fidelity)} \label{eq:F} \\
  \omega &= 1 - F
    & &\text{(drift)} \label{eq:omega} \\
  \kappa &= \sum_{i=1}^n w_i \ln c_i
    & &\text{(log-integrity)} \label{eq:kappa} \\
  \IC &= \exp(\kappa) = \prod_{i=1}^n c_i^{w_i}
    & &\text{(integrity composite)} \label{eq:IC} \\
  S &= -\sum_{i=1}^n w_i\bigl[c_i \ln c_i
    + (1-c_i)\ln(1-c_i)\bigr]
    & &\text{(entropy)} \label{eq:S}
\end{align}
These satisfy two structural identities:
\begin{align}
  F + \omega &= 1 & &\text{(duality identity)} \label{eq:duality} \\
  \IC &\leq F & &\text{(integrity bound)} \label{eq:bound}
\end{align}
The heterogeneity gap is $\Delta = F - \IC \geq 0$.
\end{definition}

The integrity bound~\eqref{eq:bound} is a Tier-1 structural
identity derived independently from Axiom-0.  The classical
AM-GM inequality emerges as a degenerate limit when channel
semantics, weights, and the guard band are stripped
away~\cite{paulus2025umcp}.

\begin{remark}[Geometric Mean Sensitivity]
$\IC$ is a weighted geometric mean.  If any single channel
$c_j = \eps$, its contribution to $\IC$ is
$\eps^{w_j} = (10^{-6})^{1/8} \approx 0.178$.
This single factor suppresses $\IC$ by $\sim 5.6\times$.
Two channels at $\eps$: $\sim 31\times$ suppression.
Four channels: $\sim 1000\times$.
The arithmetic mean $F$ is unaffected by the same magnitudes.
This asymmetry is the mechanism behind the confinement signature.
\end{remark}


% ============================================================
\section{Trace Vector Construction}\label{sec:trace}
% ============================================================

All particle data are from PDG 2024~\cite{pdg2024}.  We construct
two 8-channel trace vectors: one for fundamental particles and
one for composite hadrons.

\subsection{Fundamental Particles (17)}

Each of the 17 fundamental SM particles maps to 8 channels:
\begin{enumerate}
\item $c_1$: mass (log-normalized),
  $\log_{10}(m/m_{\rm floor}) / \log_{10}(m_{\rm ceil}/m_{\rm floor})$
\item $c_2$: $|Q/e|$ (absolute charge)
\item $c_3$: $s$ (spin, as $2s/2$)
\item $c_4$: $\log_2(N_{\rm color}+1) / \log_2(9)$ (color dof)
\item $c_5$: $(|T_3|+0.5)/1.5$ (weak isospin)
\item $c_6$: $|Y/2|/Y_{\rm max}$ (hypercharge)
\item $c_7$: $g/3$ (generation; $\eps$ for bosons)
\item $c_8$: stability (log-lifetime ratio)
\end{enumerate}
Equal weights: $w_i = 1/8$.  Guard band: $\eps = 10^{-6}$.
All channels are clamped to $[\eps, 1-\eps]$.

\subsection{Composite Particles (14 Hadrons)}

Hadrons use a different 8-channel basis, reflecting bound-state
observables:
\begin{enumerate}
\item $c_1$: mass (same normalization)
\item $c_2$: $|Q/e|$ (absolute charge)
\item $c_3$: spin
\item $c_4$: valence ($n_q/3$; mesons $= 2/3$, baryons $= 1$)
\item $c_5$: $|S|/3$ (strangeness)
\item $c_6$: $(|C|+|B'|)/2$ (heavy flavor)
\item $c_7$: stability (log-lifetime ratio)
\item $c_8$: binding fraction,
  $(\Sigma m_{\rm quarks} - m_{\rm hadron}) / \Sigma m_{\rm quarks}$
\end{enumerate}

\begin{remark}[Channel Transition at Binding]
The transition from fundamental to composite channels is itself
the structural signature of confinement.  When quarks bind, the
individually measurable quantum numbers (color, weak isospin,
hypercharge, generation) are replaced by composite observables
(valence, strangeness, heavy flavor, binding fraction).  The
channels that cannot carry information in the composite---because
the relevant quantum number is absent or cancelled---are clamped
to $\eps$.
\end{remark}


% ============================================================
\section{The Confinement Cliff}\label{sec:cliff}
% ============================================================

\subsection{Quark Kernel Values}

Table~\ref{tab:quarks} shows the kernel invariants for all six
quarks.

\begin{table}[t]
\caption{Kernel invariants for the six quarks (PDG~2024 data).
All quarks have $\IC > 0.51$; no channels near $\eps$.}
\label{tab:quarks}
\begin{ruledtabular}
\begin{tabular}{lcccccc}
\textrm{Quark} & \textrm{Gen} & $F$ & $\IC$ &
  $\kappa$ & $\Delta$ & $\omega$ \\
\colrule
up    & 1 & 0.595 & 0.561 & $-0.58$ & 0.034 & 0.405 \\
down  & 1 & 0.556 & 0.517 & $-0.66$ & 0.040 & 0.444 \\
charm & 2 & 0.662 & 0.634 & $-0.46$ & 0.029 & 0.338 \\
strange & 2 & 0.610 & 0.573 & $-0.56$ & 0.037 & 0.390 \\
top   & 3 & 0.638 & 0.589 & $-0.53$ & 0.049 & 0.362 \\
bottom & 3 & 0.667 & 0.615 & $-0.49$ & 0.052 & 0.333 \\
\colrule
Mean  &   & 0.621 & 0.581 &         & 0.040 &       \\
\end{tabular}
\end{ruledtabular}
\end{table}

Every quark has a trace vector with all 8 channels comfortably
in $(0.33, 1.0)$.  The heterogeneity gap $\Delta$ is small
($\sim 0.04$): the channels are relatively uniform.

\subsection{Hadron Kernel Values}

Table~\ref{tab:hadrons} shows the kernel invariants for all 14
hadrons, together with the number of channels at $\eps$.

\begin{table*}[t]
\caption{Kernel invariants for 14 composite hadrons.  The column
``$n_\eps$'' counts channels clamped to $\eps = 10^{-6}$.
Every hadron has $\IC < 0.029$, which is below the minimum quark
$\IC$ ($0.517$) by at least an order of magnitude.}
\label{tab:hadrons}
\begin{ruledtabular}
\begin{tabular}{llccccccc}
\textrm{Hadron} & \textrm{Content} & \textrm{Type} &
  $F$ & $\IC$ & $\kappa$ & $\Delta$ & $\omega$ & $n_\eps$ \\
\colrule
$p$        & $uud$             & Baryon & 0.550 & 0.0204
  & $-3.89$ & 0.529 & 0.450 & 2 \\
$n$        & $udd$             & Baryon & 0.395 & 0.0035
  & $-5.65$ & 0.391 & 0.605 & 3 \\
$\Lambda^0$  & $uds$           & Baryon & 0.407 & 0.0153
  & $-4.18$ & 0.392 & 0.593 & 2 \\
$\Sigma^+$   & $uus$           & Baryon & 0.526 & 0.0229
  & $-3.78$ & 0.503 & 0.474 & 2 \\
$\Xi^0$      & $uss$           & Baryon & 0.445 & 0.0045
  & $-5.41$ & 0.440 & 0.555 & 2 \\
$\Omega^-$   & $sss$           & Baryon & 0.674 & 0.0287
  & $-3.55$ & 0.645 & 0.326 & 2 \\
$\Lambda_c^+$ & $udc$          & Baryon & 0.544 & 0.0239
  & $-3.73$ & 0.521 & 0.456 & 2 \\
\colrule
$\pi^+$    & $u\bar{d}$        & Meson  & 0.476 & 0.0047
  & $-5.36$ & 0.471 & 0.524 & 3 \\
$\pi^0$    & $u\bar{u}/d\bar{d}$ & Meson & 0.334 & 0.0008
  & $-7.12$ & 0.333 & 0.666 & 4 \\
$K^+$      & $u\bar{s}$        & Meson  & 0.473 & 0.0212
  & $-3.85$ & 0.452 & 0.527 & 1 \\
$K^0$      & $d\bar{s}$        & Meson  & 0.349 & 0.0038
  & $-5.58$ & 0.345 & 0.651 & 2 \\
$J/\psi$   & $c\bar{c}$        & Meson  & 0.364 & 0.0020
  & $-6.23$ & 0.362 & 0.636 & 4 \\
$\Upsilon$ & $b\bar{b}$        & Meson  & 0.369 & 0.0008
  & $-7.09$ & 0.368 & 0.631 & 4 \\
$D^0$      & $c\bar{u}$        & Meson  & 0.317 & 0.0025
  & $-5.99$ & 0.314 & 0.683 & 2 \\
\colrule
Mean       &                   &        & 0.444 & 0.0111
  &         & 0.433 &       &   \\
\end{tabular}
\end{ruledtabular}
\end{table*}

The contrast is stark.  Quarks: $\langle \IC \rangle = 0.581$.
Hadrons: $\langle \IC \rangle = 0.011$.  A factor of
$53\times$ separating them.

\subsection{Channel Death: The Mechanism}

The IC collapse traces to specific channels being annihilated
upon binding.  Table~\ref{tab:channels} shows the trace vectors
for representative particles.

\begin{table}[b]
\caption{Selected trace vector channels for the up quark,
proton, $\pi^0$, and $J/\psi$.  Channels at $\eps = 10^{-6}$
are marked \textbf{bold}.}
\label{tab:channels}
\begin{ruledtabular}
\begin{tabular}{lcccc}
\textrm{Channel} & $u$ & $p$ & $\pi^0$ & $J/\psi$ \\
\colrule
mass\_log     & 0.627 & 0.825 & 0.762 & 0.864 \\
charge        & 0.667 & 1.000 & $\bm{10^{-6}}$ & $\bm{10^{-6}}$ \\
spin          & 0.500 & 0.500 & $\bm{10^{-6}}$ & 1.000 \\
ch.~4$^{*}$   & 0.631 & 1.000 & 0.667 & 0.667 \\
ch.~5$^{*}$   & 0.667 & $\bm{10^{-6}}$ & $\bm{10^{-6}}$ & $\bm{10^{-6}}$ \\
ch.~6$^{*}$   & 0.333 & $\bm{10^{-6}}$ & $\bm{10^{-6}}$ & $\bm{10^{-6}}$ \\
ch.~7$^{*}$   & 0.333 & 1.000 & 0.446 & 0.380 \\
ch.~8$^{*}$   & 1.000 & 0.073 & 0.799 & $\bm{10^{-3}}$ \\
\colrule
$\IC$         & 0.561 & 0.020 & 0.0008 & 0.002 \\
$\Delta$      & 0.034 & 0.529 & 0.333 & 0.362 \\
\end{tabular}
\end{ruledtabular}
{\footnotesize $^{*}$Channel 4--8 names differ between
fundamental (color, $T_3$, $Y$, gen, stability) and composite
(valence, strangeness, heavy flavor, stability, binding).}
\end{table}

For the up quark, all 8 channels are in $(0.33, 1.0)$.  The
geometric mean has no channel dragging it toward zero.

For the proton ($uud$), two channels die:
\begin{itemize}
\item \textbf{Strangeness} $= 0$: no strange quarks $\Rightarrow$
  $c_5 = \eps$.
\item \textbf{Heavy flavor} $= 0$: no charm or beauty $\Rightarrow$
  $c_6 = \eps$.
\end{itemize}
Each $\eps$ channel contributes $\frac{1}{8}\ln(10^{-6}) = -1.73$
to $\kappa$.  Two such channels push $\kappa$ from the quark range
($\sim -0.6$) down to $-3.89$.  Result:
$\IC = e^{-3.89} = 0.020$.

For $\pi^0$ ($u\bar{u}/d\bar{d}$), four channels die: charge
(neutral), spin (pseudoscalar, $s=0$), strangeness, and heavy
flavor.  Four $\eps$ contributions push $\kappa$ to $-7.12$.
Result: $\IC = 0.0008$ --- a $99.86\%$ drop from quarks.

For $J/\psi$ ($c\bar{c}$), four channels die: charge (neutral),
strangeness (no strange quarks), heavy flavor ($c + \bar{c} = 0$
net charm), and binding fraction ($m_{J/\psi} \approx 2m_c$,
so binding is negligible).  Result: $\IC = 0.002$.


% ============================================================
\section{Formal Statement and Proof}\label{sec:theorem}
% ============================================================

\begin{theorem}[Confinement as IC Collapse --- T3]\label{thm:T3}
Let $\mathcal{Q} = \{u, d, c, s, t, b\}$ denote the six quarks
and $\mathcal{H}$ the 14 composite hadrons (7 baryons + 7 mesons),
each mapped to an 8-channel trace vector with equal weights
$w_i = 1/8$ and guard band $\eps = 10^{-6}$.  Then:

\begin{enumerate}
\item \textbf{Universal suppression}: For every hadron
  $h \in \mathcal{H}$,
  \beq
    \IC_h < \IC_q^{\min}
      \equiv \min_{q \in \mathcal{Q}} \IC_q = 0.517.
  \eeq
  All 14 of 14 hadrons satisfy this bound.

\item \textbf{Collapse magnitude}: The average IC ratio is
  \beq
    \frac{\langle \IC \rangle_{\mathcal{H}}}
         {\langle \IC \rangle_{\mathcal{Q}}}
    = \frac{0.011}{0.581} = 0.019,
  \eeq
  corresponding to a $98.1\%$ collapse.

\item \textbf{Gap amplification}: The heterogeneity gap
  amplifies upon binding:
  \beq
    \frac{\langle \Delta \rangle_{\mathcal{H}}}
         {\langle \Delta \rangle_{\mathcal{Q}}}
    = \frac{0.433}{0.040} = 10.82.
  \eeq

\item \textbf{Baryon and meson ordering}: Both particle classes
  satisfy
  \beq
    \langle \IC \rangle_{\rm baryon} < \langle \IC \rangle_{\mathcal{Q}},
    \quad
    \langle \IC \rangle_{\rm meson} < \langle \IC \rangle_{\mathcal{Q}}.
  \eeq

\item \textbf{F is not annihilated}: The fidelity ratio
  \beq
    0.3 < \frac{\langle F \rangle_{\mathcal{H}}}
               {\langle F \rangle_{\mathcal{Q}}}
    = 0.715 < 1.0.
  \eeq
  Binding reduces F moderately (as expected for information
  loss), but does not destroy it.  The catastrophic collapse
  is specific to IC.
\end{enumerate}
\end{theorem}

\begin{proof}
All five claims are verified computationally against PDG~2024
data~\cite{pdg2024}, with exact values given in
Tables~\ref{tab:quarks} and~\ref{tab:hadrons}.  The total
verification comprises 19 sub-tests (1 for claim~1 as an
aggregate, 14 individual $\IC_h < \IC_q^{\min}$ checks for
claim~1, plus 4 tests for claims~2--5), all passing.

The mathematical origin of the collapse is the sensitivity
of the geometric mean to near-zero channels.  Each channel
at $\eps$ contributes $w_i \ln \eps = \frac{1}{8}\ln(10^{-6})
= -1.727$ to $\kappa$.  For $k$ dead channels,
\beq\label{eq:kappa_dead}
  \kappa = \kappa_0 + k \cdot \frac{\ln \eps}{n}
\eeq
where $\kappa_0$ is the contribution from the surviving channels.
This lowers $\IC = e^{\kappa}$ exponentially in $k$.

The physical content enters through which channels die.
Confinement erases: strangeness (for non-strange hadrons),
heavy flavor (for light hadrons), charge (for neutrals),
spin (for pseudoscalars), and sometimes binding fraction
(when $m_{\rm hadron} \approx \sum m_{\rm quarks}$).  Each
erasure contributes one $\ln(\eps)$ penalty.
\end{proof}


% ============================================================
\section{Connection to Running Coupling}\label{sec:running}
% ============================================================

The IC cliff at the quark$\to$hadron boundary connects to the
running of $\alpha_s$ via Theorem~T9 (proven
independently~\cite{umcpmetadatarepo}).  The kernel maps the
renormalization group flow to a drift trajectory:
\beq
  \omega(Q) = \frac{|\alpha_s(Q) - \alpha_s(M_Z)|}{\alpha_s(M_Z)},
  \quad F(Q) = 1 - \omega(Q).
\eeq

\begin{itemize}
\item \textbf{High $Q$ (asymptotic freedom)}: $\alpha_s \to 0$,
  $\omega$ is small, $F$ is large.  The kernel classifies this
  as \regime{Perturbative}.  Quarks with high IC live here.

\item \textbf{$Q \to \LQCD$ (confinement)}: $\alpha_s > 1$,
  $\omega$ is large, $F$ is small.  The kernel classifies this
  as \regime{NonPerturbative}.  Hadrons with collapsed IC live
  here.
\end{itemize}

The two theorems are complementary views of the same
phenomenon.  T3 sees confinement as a \emph{static signature}
(the IC cliff between two classes of particles).  T9 sees it
as a \emph{dynamic trajectory} (the drift grows as $Q$ decreases
toward $\LQCD$).


% ============================================================
\section{Why This Is Not a Restatement}\label{sec:novelty}
% ============================================================

It is essential to distinguish what the kernel provides from
what QCD already knows.

\textbf{QCD provides the mechanism (the ``why'').}
Confinement occurs because the $SU(3)$ gauge field generates
a linear confining potential at long distances.  Gluon
self-interaction (non-abelian gauge symmetry) causes the flux
tube between quarks to have constant energy per unit length
(string tension $\sigma \approx 0.18$~GeV$^2$).  This is
the dynamical explanation, rooted in a specific Lagrangian
and gauge group.

\textbf{The kernel provides the structural signature
(the ``what'').}  The IC collapse does not know about gluons,
flux tubes, or color charge.  It measures channel death: the
annihilation of specific observable properties when quarks bind.
The signature is:
\begin{itemize}
\item \textbf{Domain-independent}: any system where binding
  kills channels will show the same geometric-mean cliff.
\item \textbf{Axiom-derived}: the sensitivity of
  $\IC = \exp(\kappa)$ to zeros follows from the mathematical
  structure of the log-integrity~\eqref{eq:kappa}, which is
  a Tier-1 identity derived from Axiom-0.
\item \textbf{Falsifiable}: the prediction ``every hadron has
  $\IC < \IC_q^{\min}$'' can be tested against any new
  composite particle.  A tetraquark, pentaquark, or glueball
  with $\IC > 0.517$ would falsify Theorem~T3.
\item \textbf{Computable}: no lattice simulation required.
  The entire analysis runs in $< 1$~s on a laptop.
\end{itemize}

\begin{claim}[Scope and Complementarity]\label{claim:scope}
The kernel does not replace QCD's dynamical explanation of
confinement.  It provides something QCD currently cannot: a
quantitative, computable, domain-independent structural
signature of what confinement does to observable coherence.
The dynamics belong to a specific Lagrangian.  The structural
signature belongs to the axiom.
\end{claim}

\begin{remark}[Expandability]
The current analysis uses a static 8-channel embedding.  Within
the GCD tier structure, Tier-2 closures can host arbitrarily
rich channel sets, including dynamical observables (decay widths,
cross sections, form factors).  The confinement signature does
not depend on the number of channels; it depends on the ratio
of dead channels to total channels and the guard-band
value---both of which are frozen parameters, not tuned.
\end{remark}


% ============================================================
\section{Predictions and Falsifiability}\label{sec:predictions}
% ============================================================

Theorem~T3 generates specific, testable predictions:

\begin{enumerate}
\item \textbf{Exotic hadrons}: Tetraquarks ($qq\bar{q}\bar{q}$)
  and pentaquarks ($qqqq\bar{q}$) should have $\IC$ in the
  hadron band ($\IC < 0.03$), because they are still
  color-singlet bound states with the same dead channels
  (strangeness, heavy flavor zero for most exotics).

\item \textbf{Glueballs}: Pure gluon bound states ($gg$, $ggg$)
  predicted by QCD but not yet confirmed experimentally should
  have $\IC$ in the hadron range, because they carry no flavor
  quantum numbers at all.

\item \textbf{Quark-gluon plasma}: In a deconfined phase,
  individual quarks and gluons should recover their individual
  kernel values ($\IC > 0.5$ for quarks), and the IC cliff
  should disappear.  The phase transition should be visible as
  a discontinuity in $\langle \IC \rangle$ at $T = T_c$.

\item \textbf{Hypothetical free quarks}: If a free quark were
  ever observed (violating confinement), its trace vector
  would have no dead channels and $\IC \sim 0.55$---placing
  it on the quark side of the cliff, consistent with the
  kernel framework.

\item \textbf{New heavy hadrons}: Any undiscovered hadron with
  bottom or charm content should show the same IC collapse
  pattern, with specific dead-channel predictions based on
  its quantum numbers.
\end{enumerate}


% ============================================================
\section{The Derivation Chain from Axiom-0}\label{sec:axiom}
% ============================================================

The confinement signature is not \textit{ad hoc}.  It derives
from Axiom-0 through a specific chain:

\begin{enumerate}
\item \textbf{Axiom-0} $\Rightarrow$ collapse is generative;
  structure must return to be real.

\item \textbf{Duality identity} $\Rightarrow$ $F + \omega = 1$.
  Every process has a fidelity (what survives) and a drift
  (what is lost).

\item \textbf{Log-integrity} $\Rightarrow$
  $\kappa = \sum w_i \ln c_i$.  The logarithmic structure is
  forced by the multiplicative nature of collapse across
  independent channels.

\item \textbf{Integrity composite} $\Rightarrow$
  $\IC = \exp(\kappa)$.  This is the geometric mean of the
  trace vector---the unique composite that respects the
  multiplicative structure.

\item \textbf{Integrity bound} $\Rightarrow$ $\IC \leq F$.
  The geometric mean never exceeds the arithmetic mean.
  The gap $\Delta = F - \IC$ measures channel heterogeneity.

\item \textbf{Guard band} $\Rightarrow$ $c_i \in [\eps, 1-\eps]$.
  The guard prevents the pole at $c_i = 0$ from dominating,
  but $\eps = 10^{-6}$ is small enough that near-zero channels
  still produce massive $\kappa$ penalties.

\item \textbf{Channel death} $\Rightarrow$ when binding
  annihilates channels, $c_i \to \eps$ for those channels,
  $\kappa$ drops by $k \cdot \frac{\ln \eps}{n}$, and
  $\IC$ collapses exponentially.
\end{enumerate}

Every step is forced by the axiom or by the mathematical
structure it implies.  The confinement signature is not a
model fit, not a parameter selection, and not a
phenomenological observation imposed from outside.  It is
a structural consequence of what happens when a multiplicative
coherence measure encounters channel annihilation.


% ============================================================
\section{Conclusions}\label{sec:conclusions}
% ============================================================

We have shown that quark confinement produces a sharp,
quantitative structural signature in the GCD kernel: a $98.1\%$
collapse in the integrity composite at the quark$\to$hadron
boundary.  This is Theorem~T3, verified across 14 hadrons and
6 quarks (19/19 sub-tests, all passing).

The mechanism is \emph{channel death}: binding annihilates
specific observable channels (strangeness, heavy flavor, charge,
spin), and the geometric mean---being multiplicatively sensitive
to zeros---translates this annihilation into a two-order-of-magnitude
integrity cliff.  The arithmetic mean (fidelity) drops by only
$28.5\%$, confirming that the catastrophic collapse is specific
to the multiplicative structure of $\IC$.

This result occupies a distinct space from the QCD approach to
confinement:
\begin{itemize}
\item QCD asks \emph{why} quarks are confined (dynamics,
  Lagrangian, non-perturbative gauge theory).
\item The kernel asks \emph{what} confinement does to observable
  structure (channel death, IC cliff, gap amplification).
\end{itemize}

The kernel approach does not solve the Millennium Problem.  It
provides something complementary: a measurable, falsifiable,
computable signature that characterizes confinement's structural
consequences without requiring a non-perturbative proof.  The
signature derives from a single axiom through a traced derivation
chain, applies to any domain where binding annihilates channels,
and generates specific predictions for exotic hadrons, glueballs,
and the quark-gluon plasma phase transition.

The reference implementation, including all kernel computations,
trace vector constructions, and 19 automated tests, is available
at \url{https://github.com/calebpruett927/GENERATIVE-COLLAPSE-DYNAMICS}.

% ============================================================
\begin{acknowledgments}
C.P.\ thanks C.Pau.\ for the axiom, the frozen-contract
discipline, and the framework that makes this analysis possible.
All particle data from PDG~2024~\cite{pdg2024}.
Reference implementation:
\url{https://github.com/calebpruett927/GENERATIVE-COLLAPSE-DYNAMICS}.
\end{acknowledgments}

\bibliography{Bibliography}

\end{document}
