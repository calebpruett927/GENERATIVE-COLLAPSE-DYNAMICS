% ============================================================
% Particle Physics in the Generative-Collapse Kernel:
% Ten Tier-2 Theorems from the Standard Model
%
% Compile: pdflatex → bibtex → pdflatex → pdflatex
% ============================================================

\documentclass[
  aps,
  prd,
  twocolumn,
  superscriptaddress,
  nofootinbib,
  floatfix
]{revtex4-2}

% ── packages ──
\usepackage{amsmath,amssymb,amsthm}
\usepackage{graphicx}
\usepackage{hyperref}
\usepackage{xcolor}
\usepackage{bm}
\usepackage{booktabs}
\usepackage{multirow}

% ── theorem environments ──
\newtheorem{theorem}{Theorem}
\newtheorem{definition}[theorem]{Definition}
\newtheorem{lemma}[theorem]{Lemma}
\newtheorem{corollary}[theorem]{Corollary}
\newtheorem{remark}{Remark}

% ── UMCP macros ──
\newcommand{\tR}{\tau_{\!R}^{*}}
\newcommand{\Gam}{\Gamma}
\newcommand{\eps}{\varepsilon}
\newcommand{\tolseam}{\mathrm{tol}_{\mathrm{seam}}}
\newcommand{\dd}{\mathrm{d}}
\newcommand{\beq}{\begin{equation}}
\newcommand{\eeq}{\end{equation}}
\newcommand{\INF}{\texttt{INF\_REC}}
\newcommand{\regime}[1]{\textsc{#1}}
\newcommand{\conform}{\regime{conformant}}
\newcommand{\tvec}{\mathbf{c}}
\newcommand{\wvec}{\mathbf{w}}
\newcommand{\GM}{\mathrm{GM}}
\newcommand{\AM}{\mathrm{AM}}
\newcommand{\IC}{\mathrm{IC}}

\begin{document}

% ============================================================
% TITLE
% ============================================================
\title{Particle Physics in the Generative-Collapse Kernel:\\
Ten Tier-2 Theorems from the Standard Model}

\author{Caleb Pruett}
\email{caleb@umcp.dev}
\affiliation{UMCP Reference Implementation, GitHub}

\author{Clement Paulus}
\affiliation{UMCP / GCD / RCFT Canon}

\date{\today}

\begin{abstract}
We show that the seven-metric kernel of the Universal Measurement
Contract Protocol (UMCP) --- originally constructed for generic
computational-workflow validation --- encodes nontrivial structure
when applied to the Standard Model of particle physics.
Using PDG-tabulated quantum numbers for 17 fundamental particles
and 14 composite hadrons, we construct 8-channel trace vectors
$\tvec \in [\eps, 1-\eps]^8$ and prove ten theorems at Tier-2
(domain expansion), each verified by between 5 and 19 automated
tests (74 total, 0 failures).  The theorems connect spin-statistics
to the fidelity--drift split (T1), demonstrate strict generation
monotonicity in $F$ (T2), identify confinement as a 98\% collapse
of the integrity coefficient $\IC$ (T3), map 13 orders of magnitude
in mass to a bounded fidelity interval (T4), detect charge
quantization via $\IC$ suppression (T5), establish cross-scale
universality from femtometers to nanometers (T6), show that
electroweak symmetry breaking amplifies generation structure (T7),
verify CKM unitarity as a kernel identity with visible CP violation
(T8), reproduce asymptotic freedom as monotonic kernel flow (T9),
and recover the nuclear binding curve through kernel--binding-energy
anti-correlation (T10).  All results follow from Tier-1 identities
($F + \omega = 1$, $\IC \leq F$, $\IC = e^{\kappa}$) applied to
physically motivated trace vectors, with duality $F + \omega = 1$
verified to machine precision across all particles.
The formalism provides a Tier-2 diagnostic lens through which
Standard Model phenomenology becomes kernel-visible without
modifying the underlying physics.
\end{abstract}

\maketitle

% ============================================================
\section{Introduction}\label{sec:intro}
% ============================================================

The Universal Measurement Contract Protocol
(UMCP)~\cite{paulus2025umcp,paulus2025physicscoherence} is a
contract-first validation framework built on a single axiom:
\emph{``What Returns Through Collapse Is Real.''} Its
computational core is a seven-metric kernel that maps any
set of coherence coordinates $\tvec \in [0,1]^n$ with weights
$\wvec$ (summing to unity) to invariants $(\omega, F, S, C,
\kappa, \IC, \tau_R)$.  Three Tier-1 identities are provably
exact~\cite{paulus2026umcpcasepack}:
\beq\label{eq:tier1}
  F + \omega = 1, \quad
  \IC \leq F \;\text{(integrity bound)}, \quad
  \IC = e^{\kappa}.
\eeq

Tier-1 identities are immutable --- they hold for any input
by construction.  The question we address is: \emph{when the
inputs encode real physics, what Tier-2 structure emerges?}

We apply the kernel to the Standard Model (SM) of particle
physics, encoding each particle's quantum numbers as an
8-channel trace vector.  Ten theorems emerge, each proven
computationally against Particle Data Group (PDG) values
and verified by automated tests.  The reference implementation
is publicly available~\cite{umcpmetadatarepo}.

% ============================================================
\section{Kernel Review}\label{sec:kernel}
% ============================================================

\begin{definition}[GCD Kernel]\label{def:kernel}
Given coordinates $\tvec = (c_1, \ldots, c_n) \in [\eps, 1-\eps]^n$
and weights $\wvec = (w_1, \ldots, w_n)$ with $\sum_i w_i = 1$,
the kernel computes:
\begin{align}
  F &= \sum_i w_i \, c_i, \label{eq:F} \\
  \omega &= 1 - F, \label{eq:omega} \\
  S &= -\sum_i w_i \bigl[c_i \ln c_i + (1{-}c_i)\ln(1{-}c_i)\bigr],
  \label{eq:S} \\
  C &= \frac{1}{0.5}\,\mathrm{std}(\tvec), \label{eq:C} \\
  \kappa &= \sum_i w_i \ln(c_i + \eps), \label{eq:kappa} \\
  \IC &= \exp(\kappa). \label{eq:IC}
\end{align}
\end{definition}

The heterogeneity gap $\Delta \equiv F - \IC \geq 0$ measures channel
heterogeneity: $\Delta = 0$ if and only if all $c_i$ are equal.
This gap is the central diagnostic in what follows.

% ============================================================
\section{Trace-Vector Construction}\label{sec:trace}
% ============================================================

Each Standard Model particle is mapped to an 8-channel trace
vector $\tvec \in [\eps, 1{-}\eps]^8$ via the following channels:

\begin{table}[h]
\caption{\label{tab:channels}Eight-channel encoding for SM particles.
Each observable is normalized to $[\eps, 1{-}\eps]$.}
\begin{ruledtabular}
\begin{tabular}{clll}
$i$ & Channel & Source & Normalization \\
\hline
1 & mass\_log & $\log_{10}(m/\text{GeV})$ & Linear to $[0,1]$ \\
2 & spin\_norm & $s/s_{\max}$ & $s_{\max} = 2$ \\
3 & charge\_norm & $|Q|/Q_{\max}$ & $Q_{\max} = 2$ \\
4 & color & Color multiplicity & $\{1, 3, 8\} \to [0,1]$ \\
5 & weak\_isospin & $|I_3^W|$ & Direct \\
6 & lepton\_num & $|L|$ & Binary \\
7 & baryon\_num & $|B|$ or $B/3$ & Per quark \\
8 & generation & Gen/$3$ & $\{1,2,3\}$ \\
\end{tabular}
\end{ruledtabular}
\end{table}

Equal weights $w_i = 1/8$ are used throughout.  The guard band
$\eps = 10^{-8}$ prevents $\ln(0)$ singularities.  All 17
fundamental particles and 14 composite hadrons pass Tier-1
identities exactly.

% ============================================================
\section{Ten Theorems}\label{sec:theorems}
% ============================================================

\subsection{T1: Spin-Statistics Kernel Theorem}

\begin{theorem}[Spin-Statistics Split]\label{thm:T1}
Let $\mathcal{F}$ and $\mathcal{B}$ denote the sets of
fundamental fermions and bosons respectively.  Then
\beq
  \langle F \rangle_{\mathcal{F}} > \langle F \rangle_{\mathcal{B}},
\eeq
with a split of $0.194$ ($0.615$ vs.\ $0.421$).  Furthermore,
every quark satisfies $F_q > \langle F \rangle_{\mathcal{B}}$.
\end{theorem}

\begin{remark}
The theorem holds per-generation: within each of generations 1--3,
fermion $\langle F \rangle$ exceeds boson $\langle F \rangle$.
The split arises because fermions occupy more kernel channels
(nonzero lepton/baryon number, generation index) than bosons.
\end{remark}

\subsection{T2: Generation Monotonicity}

\begin{theorem}[Generation Ordering]\label{thm:T2}
Partition the 12 fundamental fermions by generation $g \in \{1,2,3\}$.
Then
\beq
  \langle F \rangle_{g=1} < \langle F \rangle_{g=2} < \langle F \rangle_{g=3},
\eeq
with values $0.576 < 0.620 < 0.649$.  The monotonicity holds
independently for quarks and for leptons.
\end{theorem}

\begin{remark}
This reflects the mass hierarchy: higher-generation fermions
have larger masses, which map to larger $c_1$ (mass\_log channel),
pulling $F$ upward.  The generation channel $c_8 = g/3$ reinforces
the ordering.
\end{remark}

\subsection{T3: Confinement as IC Collapse}

\begin{theorem}[IC Collapse under Binding]\label{thm:T3}
Let $\IC_q^{\min}$ be the minimum integrity coefficient among
all quarks, and let $\IC_h$ denote that of any composite hadron.
Then:
\begin{enumerate}
\item $\IC_h < \IC_q^{\min}$ for all 14 hadrons tested.
\item The average IC drops by $98.1\%$ from quarks to hadrons.
\item The heterogeneity gap amplifies by a factor of $10.8\times$ upon
      binding.
\end{enumerate}
\end{theorem}

The IC collapse is driven by the loss of the generation and
lepton-number channels in composite particles: hadrons have
$c_6 = c_8 = \eps$, which sends $\kappa \to -\infty$ via
$\ln(\eps)$.  This is the kernel signature of confinement ---
individual quark quantum numbers become invisible.

\subsection{T4: Mass--Kernel Logarithmic Mapping}

\begin{theorem}[Logarithmic Compression]\label{thm:T4}
The 17 fundamental particles span $13.2$ orders of magnitude
in mass.  The kernel maps this range to $F \in [0.37, 0.73]$.
Among quarks, the Spearman rank correlation between mass and
$F$ is $\rho = 0.77$.
\end{theorem}

The logarithmic compression is a consequence of the $\log_{10}$
normalization in channel 1.  The bounded output is a feature,
not a bug: the kernel measures \emph{coherence heterogeneity},
not magnitude.

\subsection{T5: Charge Quantization Signature}

\begin{theorem}[Neutral Suppression]\label{thm:T5}
Neutral particles (photon, gluon, $Z^0$, neutrinos) have
$\IC \ll \IC_{\text{charged}}$.  The ratio
\beq
  \frac{\langle \IC \rangle_{\text{neutral}}}
       {\langle \IC \rangle_{\text{charged}}} = 0.020
\eeq
represents a $50\times$ suppression.  The photon has
$\IC = 7.6 \times 10^{-4}$, the smallest of any fundamental
particle.
\end{theorem}

\begin{remark}
Neutral particles carry $c_3 = \eps$ (charge channel near zero).
Since $\IC = \exp\!\bigl(\sum w_i \ln c_i\bigr) = \prod c_i^{w_i}$
is a weighted geometric mean, even one channel near $\eps$
catastrophically suppresses $\IC$.  This is the integrity bound
at work: neutrality creates maximal channel heterogeneity.
(The classical AM-GM inequality is the degenerate limit when
the collapse field is removed.)
\end{remark}

\subsection{T6: Cross-Scale Universality}

\begin{theorem}[Scale Ordering]\label{thm:T6}
Let $\langle F \rangle$ denote the mean fidelity for three
particle classes --- fundamental (fm scale), atomic (pm scale),
and composite (fm scale, bound).  Then
\beq
  \langle F \rangle_{\text{composite}} <
  \langle F \rangle_{\text{atomic}} <
  \langle F \rangle_{\text{fundamental}},
\eeq
with values $0.444 < 0.516 < 0.558$.
\end{theorem}

The same kernel, applied at three different length scales
($10^{-15}$~m, $10^{-12}$~m, $10^{-10}$~m), produces a
consistent ordering: binding reduces fidelity, and composite
systems sit below their constituents.

\subsection{T7: Symmetry Breaking as Trace Deformation}

\begin{theorem}[EWSB Amplification]\label{thm:T7}
Electroweak symmetry breaking (EWSB) enters the kernel through
the mass\_log channel via Yukawa couplings
$y_f = \sqrt{2}\,m_f / v$ ($v = 246.22$~GeV).  Define the
generation spread $\sigma_g \equiv \max_g \langle F \rangle_g
- \min_g \langle F \rangle_g$.  Then the broken theory has
\beq
  \sigma_g^{\text{broken}} = 0.073 > \sigma_g^{\text{unbroken}} = 0.046.
\eeq
Higher generations gain more fidelity from the Higgs:
$\Delta F_{g=3} > \Delta F_{g=2} > \Delta F_{g=1}$.
\end{theorem}

Before EWSB, all fermions share a common mass channel
($c_1 = 0.5$); after EWSB, Yukawa couplings differentiate
them.  The kernel makes the symmetry breaking visible as a
monotonically increasing $\Delta F$ per generation.

\subsection{T8: CKM Unitarity as Kernel Identity}

\begin{theorem}[CKM in the Kernel]\label{thm:T8}
Each row of the CKM matrix $V_{\text{CKM}}$, treated as a
3-channel trace vector $(|V_{ij}|^2)_{j=1}^3$, passes all
Tier-1 identities.  Furthermore:
\begin{enumerate}
\item Row~1 has a larger heterogeneity gap than Row~2, because
      $|V_{ub}|^2 \approx 10^{-5}$ acts as an extreme
      zero-channel.
\item The Jarlskog invariant $J_{\text{CP}} = 3.0 \times 10^{-5}$
      (CP violation) is kernel-visible.
\end{enumerate}
\end{theorem}

The Wolfenstein parametrization
($\lambda = 0.2257$, $A = 0.814$, $\rho = 0.135$, $\eta = 0.349$)
generates rows whose $\sum |V_{ij}|^2$ departs from unity by
$O(\lambda^4) \approx 0.002$.  The kernel classifies this as
the ``Tension'' regime --- appropriate for an approximation
at third order.

\subsection{T9: Running Coupling as Kernel Flow}

\begin{theorem}[Asymptotic Freedom]\label{thm:T9}
The one-loop QCD running coupling
\beq
  \alpha_s(Q^2) = \frac{\alpha_s(M_Z^2)}
    {1 + \frac{b_0 \alpha_s(M_Z^2)}{2\pi}
      \ln\!\bigl(\tfrac{Q^2}{M_Z^2}\bigr)}
\eeq
with $b_0 = 11 - \tfrac{2}{3}n_f = 7$ is monotonically
decreasing for $Q \geq 10$~GeV.  At $Q = M_Z$,
$\alpha_s = 0.118$ (perturbative).  At $Q = 0.5$~GeV, the
formula exceeds unity, signaling confinement (``NonPerturbative''
regime).
\end{theorem}

The kernel provides a natural regime classification:
$\alpha_s < 0.3$ is perturbative (Stable); $0.3 \leq \alpha_s < 1$
is transitional (Watch); $\alpha_s \geq 1$ is nonperturbative
(Collapse).  Asymptotic freedom maps to a monotonic flow from
Stable to Collapse as $Q$ decreases.

\subsection{T10: Nuclear Binding Curve Correspondence}

\begin{theorem}[Binding--Gap Anti-correlation]\label{thm:T10}
Using the Bethe--Weizs\"acker semi-empirical mass formula to
compute binding energy per nucleon $B/A$ for elements
$Z = 1, \ldots, 118$, and the 12-channel cross-scale kernel to
compute the heterogeneity gap $\Delta$:
\begin{enumerate}
\item $r(B/A,\,\Delta) = -0.41$ (Pearson anti-correlation).
\item $B/A$ peaks at $Z \in [23, 30]$ (Cr--Zn region),
      consistent with the Fe/Ni experimental peak.
\item Nuclear magic numbers ($Z \in \{2, 8, 20, 28, 50, 82\}$)
      are detectable through the magic-proximity channel.
\end{enumerate}
\end{theorem}

The anti-correlation means: elements with higher binding energy
per nucleon have \emph{smaller} heterogeneity gaps --- they are more
coherent in the kernel sense.  Iron-group nuclei sit at both
the binding-energy maximum and a local gap minimum.

% ============================================================
\section{Duality and Consistency}\label{sec:duality}
% ============================================================

The duality identity $F + \omega = 1$ is verified to machine
precision ($\max|\delta| = 0$) across all 17 fundamental
particles, confirming that the trace-vector construction
preserves Tier-1 exactly.

All 74 individual tests pass with zero failures.  The
complete test suite runs in under 200~ms on commodity
hardware, making the formalism suitable for continuous
integration.

% ============================================================
\section{Tier-2 Diagnostic Interpretation}\label{sec:interpretation}
% ============================================================

These ten theorems demonstrate that the GCD kernel functions
as a \emph{diagnostic lens} for particle physics, not a
replacement for it.  The kernel does not predict masses or
coupling constants; instead, it provides a uniform language
in which existing SM results become visible as kernel patterns:

\begin{itemize}
\item \textbf{Spin-statistics} (T1) appears as a fidelity split.
\item \textbf{The mass hierarchy} (T2, T4) appears as generation
  monotonicity and logarithmic compression.
\item \textbf{Confinement} (T3) appears as IC collapse --- the
  loss of individual quantum numbers upon binding.
\item \textbf{Charge quantization} (T5) appears as $\IC$
  suppression via the integrity bound ($\IC \leq F$).
\item \textbf{Universality} (T6) appears as scale-ordered fidelity.
\item \textbf{Symmetry breaking} (T7) appears as trace deformation
  amplifying generation structure.
\item \textbf{Flavor mixing} (T8) appears as CKM rows in the
  kernel, with CP violation as a measurable Jarlskog invariant.
\item \textbf{Asymptotic freedom} (T9) appears as monotonic
  coupling flow within kernel regimes.
\item \textbf{Nuclear stability} (T10) appears as binding--gap
  anti-correlation, with magic numbers as detectable features.
\end{itemize}

The value of this lens is threefold:
(1)~it provides a \emph{single diagnostic framework} spanning
subatomic, atomic, and nuclear scales;
(2)~it connects disparate phenomena (confinement, EWSB, CKM mixing)
through the common algebra of the heterogeneity gap;
(3)~it is computationally cheap and fully automated, enabling
continuous monitoring via CI pipelines.

% ============================================================
\section{How to Use It: Practical Guide}\label{sec:howto}
% ============================================================

The formalism is implemented as a standalone Python module
(\texttt{particle\_physics\_formalism.py}) in the UMCP
repository~\cite{umcpmetadatarepo}.

\subsection{Running the Theorems}

\begin{verbatim}
pip install -e ".[all]"
python closures/standard_model/\
  particle_physics_formalism.py
\end{verbatim}

Output: 10/10~PROVEN, 74/74 tests, runtime $< 200$~ms.

\subsection{Adding a New Particle}

To extend the formalism to a new particle (e.g., a BSM candidate):

\begin{enumerate}
\item Add the particle to the \texttt{FUNDAMENTAL\_PARTICLES}
  dictionary in \texttt{subatomic\_kernel.py} with its
  quantum numbers.
\item Run \texttt{particle\_physics\_formalism.py} ---
  existing theorems automatically include the new entry.
\item If needed, add new theorems as functions
  \texttt{theorem\_T11()}, etc., following the
  \texttt{TheoremResult} dataclass pattern.
\end{enumerate}

\subsection{Reading the Output}

Each theorem produces a structured result:

\begin{verbatim}
T3 Confinement as IC Collapse
  Statement: Binding quarks into hadrons
    collapses IC by >90%
  Tests: 19/19 PASSED
  Verdict: PROVEN
  Details:
    quark_avg_IC = 0.0225
    hadron_avg_IC = 0.000434
    ic_collapse_pct = 98.07%
    gap_amplification = 10.82x
\end{verbatim}

The verdict is PROVEN when all sub-tests pass, FAILED otherwise.
This maps directly to UMCP's three-valued logic:
PROVEN $\to$ \conform{}, FAILED $\to$ \regime{nonconformant}.

% ============================================================
\section{What It Looks Like: Reality Through the Kernel}
\label{sec:reality}
% ============================================================

The kernel provides a specific \emph{perceptual frame} for
particle physics.  Through this lens:

\textbf{The Standard Model is a fidelity landscape.}
Each particle lives at a point $(F, \IC, \Delta)$ in kernel
space.  Fermions cluster at high $F$ ($\sim 0.6$); bosons at
lower $F$ ($\sim 0.4$).  The photon sits at the extreme:
$F = 0.37$, $\IC = 7.6 \times 10^{-4}$, $\Delta = 0.37$ ---
it is the most heterogeneous fundamental particle, with most
channels near $\eps$.

\textbf{Confinement is visible as a cliff.}  The $\IC$ landscape
drops by two orders of magnitude at the quark$\to$hadron
boundary.  This cliff is not put in by hand --- it emerges from
the loss of generation and lepton-number channels when quarks
bind into color-singlet states.

\textbf{The mass hierarchy is a slope.}  Plotting $F$ against
generation number yields a monotonically rising curve.  The
slope is steeper for quarks than leptons, reflecting the
stronger mass splitting in the quark sector.

\textbf{Symmetry breaking is a deformation.}  Before EWSB,
all fermions have the same mass channel ($c_1 = 0.5$).
After EWSB, Yukawa couplings spread the mass channel,
and the generation slope steepens.  The kernel makes this
transition quantitatively visible: $\sigma_g$ increases
from $0.046$ to $0.073$.

\textbf{CP violation is a gap asymmetry.}  The CKM matrix,
viewed as three kernel trace vectors (one per row), shows
that Row~1 has a larger gap than Row~2.  The physical origin
is $|V_{ub}| \ll |V_{cb}|$: the extreme smallness of $V_{ub}$
creates maximal heterogeneity in Row~1.  The Jarlskog
invariant $J = 3.0 \times 10^{-5}$ quantifies the CP
violation that accompanies this asymmetry.

\textbf{Nuclear stability is an anti-correlation.}  The most
tightly bound nuclei (iron group) have the smallest heterogeneity
gaps.  Moving away from the iron peak in either direction
increases the gap --- lighter nuclei are surface-dominated
(less coherent), heavier nuclei are Coulomb-dominated
(less coherent).  Magic-number nuclei appear as local
dips in the gap landscape.

This is what the Standard Model ``looks like'' through the
GCD kernel: a structured landscape of fidelity, integrity,
and gap, where known physics manifests as geometric features
(slopes, cliffs, dips, asymmetries) that are computable,
testable, and continuous-integration--ready.

% ============================================================
\section{Conclusions}\label{sec:conclusions}
% ============================================================

We have demonstrated that the GCD kernel, a domain-agnostic
diagnostic tool designed for computational workflow validation,
produces nontrivial and physically meaningful structure when
applied to the Standard Model.  Ten theorems, each testable
and reproducible, connect kernel observables ($F$, $\IC$,
$\Delta$) to established SM phenomena.

No new physics is introduced.  The kernel acts as a
\emph{translation layer}: it does not explain why the top
quark is heavy or why $|V_{ub}|$ is small, but it provides
a uniform diagnostic language in which these facts become
kernel-visible, testable, and comparable across scales
(subatomic $\to$ atomic $\to$ nuclear).

Future directions include:
\begin{enumerate}
\item BSM sensitivity: can the kernel detect deviations
  from SM predictions in new particles or anomalous couplings?
\item Higher-order embeddings: channel counts beyond 8,
  incorporating decay widths, lifetimes, or form factors.
\item Dynamic kernel flow: tracking kernel invariants as
  functions of energy scale $Q$, connecting to renormalization
  group evolution.
\item Experimental anchoring: using measured cross sections
  and branching ratios as direct kernel inputs, bypassing
  the theoretical trace-vector construction.
\end{enumerate}

The reference implementation, including all ten theorems
and 74 automated tests, is available at
\url{https://github.com/calebpruett927/GENERATIVE-COLLAPSE-DYNAMICS}.

% ============================================================
\begin{acknowledgments}
C.P.\ thanks C.Pau.\ for the frozen-contract discipline and
the axiom that grounds this work.  Reference implementation:
\url{https://github.com/calebpruett927/GENERATIVE-COLLAPSE-DYNAMICS}.
\end{acknowledgments}

\bibliography{Bibliography}

\end{document}
